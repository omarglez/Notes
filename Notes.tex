\documentclass[12pt,a4paper]{report}
\usepackage[latin1]{inputenc}
\usepackage{amsmath}
\usepackage{amsfonts}
\usepackage{amssymb}
\usepackage{graphicx}
\usepackage[left=2.00cm, right=2.00cm, top=2.00cm, bottom=2.00cm]{geometry}
\author{Omar Gonzalez}
\begin{document}

\section*{Introduction Week 1 Jan 5}
\subsection*{What is Structural Geology?}
Observation-driven discipline that studies the deformation of solid materials (rocks and ices) in Earth and other planets.
It is used to solve civil engineering problems as well as mining ones.
Studies brittle deformation and creep. The former with localized discontinuity (discontinuous in space and time) and the latter with volume or surface deformation and continuous over some timescale.

\subsection*{Types of Data}
Field observations (outcrop-map scale)\\
\indent Tools: compass, hammer, hand lens, notebook, basemap or GPS for location.\\
\indent Tools 2: camera $\rightarrow$ photogrammetry. LiDAR $\rightarrow$ light detection and ranging, terrestiral (TLS) $\approx$ few km - mm resolution. Airborne - m-pixels any areas, expensive. Satellite - 10's m in N. Am - global 90m pixels. INSAP - differenc of 2 radar images.
GPS - campaign - long-term motion (annual averages), tectonic plate rates, creeping landslides, ground water withdrawal.
    - continuous

Thrust (dip \verb|<| 30) or Reverse faults are the opposite of a normal fault where the bedding "goes down".

\section*{Stereonets Week 2 Jan 12}
Every plane and every line intersect at the same point (center of a sphere).
Structural geologists use "Equal Area" or Schimt net.
Planes are drawn from the bearing point, as measured by rotating to the horizontal axis, through the great arc which is at a distance dip from
the respective dip orientation. Poles are points located 90 degrees further. Intersection points are aligned with the vertical to mark the trend,
while the plunge is given by its distance to the horizontal extreme points.

\subsection*{Geologic Map Analysis}
Cambrian rocks settle in layers, Ordovician rock is an intrusive dyke and the cretaceous one is not horizontal.
Ks is not horizontal or vertical but it is pretty thick. An angular unconformity can be seen in the cretaceous rock at the intersection with rhyolite.
The diabase straight lines are characteristic to a vertical plane (moderate dip contact).
Cambrian rocks seem to be parallel and very steep but not parallel.
The dike indicates an apparent left lateral offset.\\
\\
By tracing two strike line on the Ks, we see they are more or less parallel which would indicate a planar contact.
Distance between strike lines at 330 and 400m is around 2.5cm so 125m.
125m, 100m depth gives 38 degrees of true dip.
The unconformity is tilted post k deformation.
Cq has an apparent right lateral offset, opposite to the dyke one.
Vertical offsets can create the illusion of lateral offsets, according to the tilting of the contact.
Assuming the fault is a plane, and by drawing the strike lines at 100m and at 300m.
We measure the dip as before, and get 73 degrees.\\
\\
\textbf{Fault}: A fracture across which there has been shear motion.\\
\textbf{Unconformity}: Contact between two rock units of contrasting orientation, age or origin.\\
\textbf{Angular Unconformity}: Difference in orientation between young and old units.\\
\textbf{Disconformity}: Rock unit origins are not consistent for parallel sediments.\\
\textbf{Nonconformity}: Different types of rock together, that would usually not be there.

\section*{Map Analysis Week 3 Jan 19}
Identifying youngest rocks by their position relative to the other contacts.
We will use a Piercing point to measure the displacement of the fault in the map.\\
\\
\textbf{Piercing Point}: A linear feature offset by a fault which enables true displacement to be measured.\\
\\
In this case, we use the intersection of two planes (the sandstone bed and the quartz porphyry).
We trace the line's trend on both sides of the fault, and then find the points where they intersect with the fault strike lines at the same elevation.
The true displacement line is on the fault plane.\\
\\
Next Map\\
Older rocks surrounded by younger is representative of a layer folding eroding model.\\
Youngest:\\
Qls: landslide deposits\\
-angular unconformity\\
Tv: Teriary volcanics\\
-angular unconformity\\
-folding\\
Cretaceous (k)\\
Jurassic\\
-disconformity
Pennsylvanian (Carboniferons)\\
Mississipian (Carboniferons)\\
Devonian\\
Ordivician
Cambrian (C)
pre-Cambrian (pC)

%625 vs 85 m,  754m horizontal distance, 320 bearing, 36 plunge, 930 true displacement

% photo 1 sedimentary conglomerate on top. Beds of alternating rocks (sands and shales), secondary structures: faults on the sides


\section*{Folds Week 4 Jan 26}
The hinge line can be symmetric or asymmetric, and the extension of the planes in the folds around it will lead to the same hinge line.
The side are called limbs. If the hinge line is strait, then the fold is cylindrical.
A trace is the line of intersection between a plane and your plane of view.
An anticline and a syncline are the opposing "noses" of the fold, where the anticline is on top.\\
\\
An "ideal" fold (two planes intersecting) is called a chevron fold.
Chevron folds can accommodate multiple folds of the same size, while concentric folds must get smaller.
Folds can be harmonic were that wavelength and amplitude remain constant, or disharmonic, when they are not.

\subsection*{Axial Plane Dips}
\verb|~|90 = upright\\
\verb|~|60 = inclined\\
enough to overturn a limb = overturned\\
nearly horizontal = recumbent

\subsection*{Parallel Folds}
These are thickened in the hinge and require a flow of rock.
\textbf{Parasitic} folds are folds that lie on top of the limb of another fold.
Their hinge trend on the same direction as the limb's hinge.
They are composed of z-folds (going up), s-folds (going down) and m-folds (horizontal).
The \textbf{vergence} direction (where the parasitic fault is going) is given by its short vs long limbs.

Dip isogons are lines connecting points in the fold (inner and outer surfaces) where the tangent line has the same plunge.
The pattern of isogons defines the classification of the fold.\\
\\
\subsection*{Mechanisms of Folding}
Buckling is compression from both sides towards the center of the planar contact.\\
Passive Folding is similar to buckling, but with a rotational component, so it folds naturally.\\
Bending is when the deformation is perpendicular to the contact effectively pushing it in the "middle".\\
\\
No thickening in the hinge is consistent with the material being stiff.
The class of fold in the layers varies from one to the other, so the stiffer layer makes the other one thicker in the hinges and slimmer in the limbs.
One example of this is alternating shale and sandstone layers.
Parasitic folds are born from slim stiff layers folding with a smaller wavelength than the surrounding, thicker layer.\\
\\
A series of harmonic layers are an indication of passive folding.
Bending can occur from forceful intrusions, reactive faults and between boudins (shattered/thinned layer pieces from a stiff material that broke when extending). In the case of a reactive fault, it can be a blind fault (not apparent in the surface) that pushes the layers above it into a monocline.
When you have a series of thin, stiff layers with little friction between them, they can slip when folded (like pages in a phone book).
This can also happen if the layers are separated by very thin weak layers. This is called \textbf{flexural slip}.\\
\\
\textbf{Flexural flow} is similar but involves a bit of flow within the material.
\textbf{Kink bands} are in little monocline folds separating the straight sides.
When looked at from far enough, it may look like a fault.
\textbf{Sheath folds} are extremely non-cylindrical folds that push so far as to create a space inside it.
\section*{Displacement Week 5 Feb 2}
Displacement from homogeneous deformation in an object can be represented by a displacement field.
In heterogeneous deformation there are different particle paths for different points in the object.
The strain and stretch are defined as follows:
$$ e = \frac{\Delta l}{l_0} = \frac{l - l_0}{l_0} $$
$$ S = 1 + e = \frac{l}{l_0} $$
When there is stretch on an inclined contact, you get not only fragmentation, but also an angular shear $\psi$.
The angular shear can be measured with respect to different points in the object, or different orientations.
\textbf{Simple shear} involves a single rotational axis involved in the deformation, while \textbf{pure shear} means there is no rotation on any axis.
\subsection*{Rheology}
Rheology is the relation between stress and strain.
$$ stress = \frac{force}{area} = \frac{m \cdot \vec{a}}{\pi r^2} = \frac{mass(water + beaker) \cdot 9.81}{\pi r^2} \frac{kg m/s^2}{m^2} (Pa)$$
Applying stress onto a cylinder (lie for the lab) will begin with recoverable (elastic) deformation.
This results in a linear proportion, $ \sigma = Ee $ where $E$ is Young's modulus and $\sigma$ is the stress.
After the deformation reaches the elastic limit, we can get \textbf{ductile} deformation (homogeneous and distributed) or \textbf{brittle} deformation where the fracture is localized.\\
\\
We will use the linear (Newtonian) viscosity relation between stress and strain rate today.
$$ \sigma = viscosity \cdot \frac{strain}{time} $$
Where the units of viscosity are $Pa \cdot s$.
The equation we will use in the lab will be as follows:
$$ \sigma = Ee + n\mathring{e} $$
\\
Coaxial vs non-coaxial deformation. With coaxial, the axes of shortening have elongations constant for all increments of strain. 
\section*{Foliations and Lineations Week 6 Feb 9}
Rocks may have a tendency to break along their foliations.
This characteristic is called cleavage.
Foliations are alignment of the minerals within the rock.\\
\\
Recap: 3 types of deformation.
Some materials follow more than one type of deformation, like Maxwell solids.
Some fabrics form through shear deformation, and the change from the regular non-elongated fabric to the elongated one defines whether the motion was left-lateral or right-lateral.
\subsection*{Crystal Defects}
\textbf{Point Defect} is a 0-dimensional fault in a crystal.
There are 3 types of point defects: vacancy, substitution and interstitial, where one atom is either missing, been replaced, or intruding in the material respectively.
\textbf{Line Defect} are abundant when crystals form and easy to produce when they deform.
There are two types: edge dislocation and screw dislocation.
Edge dislocation is when "layers" don't match, while screw dislocation is when the crystal is not connected in a grid fashion, and connects in a spiral way.\\
\\
Removing defects in the structure of materials can make the material stronger, which is what is done with glass, heating it so the energy allows deformations to move to the exterior of the material, balancing the bonds between the atoms.
Vacancies can migrate too (diffusion creep), although it is harder for point defects to do so.
With line defects, a lower temperature brings a dislocation which glides along the planes of the crystal, a medium temperature allows for it to moved across different glide planes, and at higer temperatures we obtain volume diffusion creep (annealing) which brings the most stable rocks.
When edge dislocations accumulate, if they move and group into a line, we call this a subgrain boundary, which is where the crystal changes orientation. We can observe this because the amount of light that goes through the crystal depends on the orientation of the lattice, so subgrain boundaries show up with a different shade.

\end{document}
